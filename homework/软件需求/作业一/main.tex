\documentclass[11pt, a4paper, oneside]{ctexbook}
\usepackage{amsmath, amsthm, amssymb, bm, graphicx, hyperref, mathrsfs, enumitem, geometry, listings, xcolor}
\title{{\Huge{\textbf{《中南大学\ 软件测试工程》}}}\\课后作业一}
\author{徐鸣飞}
\date{2023 年 10 月 20 日}
\linespread{1.5}

\newtheorem{theorem}{定理}[section]
\newtheorem{definition}[theorem]{定义}
\newtheorem{lemma}[theorem]{引理}
\newtheorem{corollary}[theorem]{推论}
\newtheorem{example}[theorem]{例}
\newtheorem{proposition}[theorem]{命题}

\geometry{a4paper,scale=0.7}


\begin{document}

\maketitle
\pagenumbering{roman}
\setcounter{page}{1}
\newpage
\pagenumbering{Roman}
\setcounter{page}{1}
\tableofcontents
\newpage
\setcounter{page}{1}
\pagenumbering{arabic}

\chapter{基础概念}
\section{软件功能需求}
\begin{definition}
    功能需求是软件系统所必须具备的功能或服务,涉及系统能够执行的任务、操作或功能。这些需求描述了系统应该如何响应特定的输入,以及在特定的情况下应该产生怎样的输出。
\end{definition}
下面是一些常见的软件功能需求的案例:
\begin{enumerate}[itemsep=10pt,parsep=0pt,partopsep=0pt,topsep=0pt]
    \item 电子商务网站:
    
    \begin{itemize}[itemsep=0pt,parsep=0pt,partopsep=0pt,topsep=0pt]
        \item 用户注册和登录
        \item 商品搜索和浏览
        \item 购物车管理和结算
        \item 订单跟踪和配送管理
        \item 用户评价和反馈
    \end{itemize}
    \item 医疗保健软件:
    
    \begin{itemize}[itemsep=0pt,parsep=0pt,partopsep=0pt,topsep=0pt]
        \item 病历管理和患者档案
        \item 预约挂号和排队管理
        \item 医生处方和药品发放
        \item 疾病诊断和治疗方案管理
        \item 医疗报告和检测结果管理
    \end{itemize}
    \item 学校管理系统:
    
    \begin{itemize}[itemsep=0pt,parsep=0pt,partopsep=0pt,topsep=0pt]
        \item 学生信息管理和档案管理
        \item 课程安排和教师安排
        \item 成绩管理和考试安排
        \item 学生选课和课程评价
        \item 教职工通知和沟通平台
    \end{itemize}
\end{enumerate}

\section{软件非功能需求}
\begin{definition}
    非功能需求是指软件系统除了功能之外的特定要求,包括性能、安全性、可靠性、可用性、可维护性等方面的要求。这些需求通常涉及到系统运行的环境、用户体验以及系统本身的特定属性。
\end{definition}
下面是一些常见的软件非功能需求的案例:
\begin{enumerate}[itemsep=10pt,parsep=0pt,partopsep=0pt,topsep=0pt]
    \item 性能需求
    
    \begin{itemize}[itemsep=0pt,parsep=0pt,partopsep=0pt,topsep=0pt]
        \item 响应时间:系统必须在用户发出请求后的2秒内响应。
        \item 吞吐量:系统应该能够同时处理1000个并发用户。
        \item 资源利用率:系统在运行时不应占用超过50\%的服务器资源。
    \end{itemize}
    \item 安全性需求
    
    \begin{itemize}[itemsep=0pt,parsep=0pt,partopsep=0pt,topsep=0pt]
        \item 数据加密:系统必须采用128位AES加密算法对用户的敏感数据进行加密。
        \item 用户身份验证:系统应该要求用户使用双因素身份验证登录。
        \item 访问控制:系统管理员应该有权限控制对系统的访问权限。
    \end{itemize}
    \item 可靠性需求:
    
    \begin{itemize}[itemsep=0pt,parsep=0pt,partopsep=0pt,topsep=0pt]
        \item 故障恢复:系统应该能够在发生故障后30分钟内恢复正常运行。
        \item 数据完整性:系统中的数据应该具有完整性,不能出现数据丢失或损坏的情况。
        \item 容错能力:系统应该能够自动处理系统错误,并避免导致系统崩溃。
    \end{itemize}
    \item 可维护性需求:
    
    \begin{itemize}[itemsep=0pt,parsep=0pt,partopsep=0pt,topsep=0pt]
        \item 可扩展性:系统应该能够轻松扩展以适应未来的增长需求。
        \item 可测试性:系统的各个模块应该易于单独测试和调试。
        \item 文档完整性:系统的相关文档应该详尽完整,以便未来维护和更新。
    \end{itemize}
    \item 可用性需求:
    
    \begin{itemize}[itemsep=0pt,parsep=0pt,partopsep=0pt,topsep=0pt]
        \item 用户界面友好性:用户界面应该简洁易用,符合用户的直觉习惯。
        \item 多语言支持:系统应该支持多种语言,以满足不同国家和地区的用户需求。
    \end{itemize}
\end{enumerate}

\chapter{软件非功能需求反例}
接来下将介绍一些由于未实现软件非功能需求而导致事故的反例:
\begin{enumerate}
    \item 可靠性故障反例:2017年,亚马逊云服务发生故障(AWS云存储服务S3出现大宕机,原因是一名程序员运行了一个错误脚本,结果输错了一个字母,导致大量服务器被删),导致包括网站、应用程序和物联网设备在内的大量服务受到影响。这一事件严重影响了许多企业的运营,暴露了未能满足可用性需求可能带来的风险。\footnote{可靠性是指系统保持正常运行和可访问的能力,是非功能性需求中的一个重要方面。在这种情况下,由于S3服务中断,许多企业和服务无法正常访问或提供服务,导致了严重的业务中断和影响。这突显了云服务的可靠性对企业和消费者的重要性,同时也强调了保持高可靠性的挑战和复杂性。}
    \item 安全性需求反例:Equifax是一家信用评级机构,2017年因安全漏洞而遭受了严重的数据泄露事件。攻击者利用漏洞获取了超过1亿用户的敏感信息,包括社会安全号码和信用卡信息。黑客于当年5月到7月间利用网络安全漏洞入侵Equifax系统,导致1.47亿人信息泄露,其中包括姓名、地址、出生日期、身份证号以及护照、驾照、信用卡信息等,美国、英国、加拿大等多国公民受到影响。
    \item 性能需求反例:2012年,美国纳斯达克证券交易所发生了一次严重的交易系统故障,很多证券交易所的交易员未能及时对订单进行确认,导致了市场参与者在数小时、甚至几天的时间内都无法获知他们持有Facebook股票的风险,并导致数十亿美元的交易被取消或延迟。
    \item 可维护性需求:1998年的火星探测器失联事件,部分原因是由于软件中的错误导致探测器未能正确执行指令,但由于代码的复杂性和缺乏适当的文档,NASA难以及时发现和解决问题。
    \item 可用性需求反例:一些科学研究领域的软件由于复杂的算法和模型,导致了软件的可维护性和可理解性较差。尽管这些软件具有强大的分析功能,但由于缺乏良好的文档和模块化设计,使得其他研究人员难以理解和修改这些软件,从而限制了其进一步的应用和发展。
\end{enumerate}
\end{document}