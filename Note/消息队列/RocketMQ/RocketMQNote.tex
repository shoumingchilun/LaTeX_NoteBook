\documentclass[11pt, a4paper, oneside]{ctexbook}
\usepackage{amsmath, amsthm, amssymb, bm, graphicx, hyperref, mathrsfs, enumitem, geometry, listings, xcolor, listings, fontspec, caption}

% 标题、作者、创建日期
\title{{\Huge{\textbf{RocketMQ}}}\\学习笔记}
\author{寿命齿轮}
\date{2024年1月20日}

% 设置全局字体
\setmainfont{Times New Roman}
\let\kaishu\relax                               %清除旧定义
\newCJKfontfamily\kaishu{KaiTi}[AutoFakeBold]   %重定义\kaishu,开启加粗功能

% 设置页面的尺寸和布局。
\geometry{a4paper,scale=0.75}

% 行间距为1.5倍
\linespread{1.5}

% 生成的书签大纲包含章节序号
\hypersetup{
  bookmarksnumbered=true
}

% 定义定理环境
\newtheorem{theorem}{定理}[chapter]
\newtheorem{definition}[theorem]{定义}
\newtheorem{lemma}[theorem]{引理}
\newtheorem{corollary}[theorem]{推论}
\newtheorem{example}[theorem]{例}
\newtheorem{proposition}[theorem]{命题}

% 自定义配置
% 设置全局的 enumerate 环境项之间的距离
\setlist[enumerate]{itemsep=5pt, parsep=0pt, leftmargin=20pt, topsep=5pt, partopsep=0pt}

% 定义新环境
% 定义颜色
\definecolor{commentcolor}{RGB}{182,73,1}
% 定义代码
\lstnewenvironment{java}[1][]{
  \lstset{
    language=Java,
    basicstyle=\ttfamily,
    keywordstyle=\color{blue},
    commentstyle=\color{green!60!black},
    stringstyle=\color{commentcolor},
    showstringspaces=false,
    breaklines=true,
    frame=single,
    flexiblecolumns=true,
    backgroundcolor=\color{gray!5},
    numbers=left,
    numberstyle=\tiny,
    #1
  }
}{}
\lstnewenvironment{plsql}[1][]{
  \lstset{
    language=SQL,
    morekeywords={BEGIN,DECLARE,END,IF,ELSE,ELSIF,LOOP,WHILE,PROCEDURE,FUNCTION},
    basicstyle=\ttfamily,
    keywordstyle=\color{blue},
    commentstyle=\color{green!60!black},
    stringstyle=\color{commentcolor},
    showstringspaces=false,
    breaklines=true,
    frame=single,
    flexiblecolumns=true,
    backgroundcolor=\color{gray!5},
    numbers=left,
    numberstyle=\tiny,
    #1
  }
}{}

% 文章开始
\begin{document}

% 生成标题
\maketitle

% 生成作者的话
\newpage                    %新的一页
\pagenumbering{roman}       %页码以小写罗马数字形式表示
\setcounter{page}{1}        %设置当前页为第一页
\section*{作者的话}
本文用于记录作者在学习RocketMQ过程中所记录的笔记。

学习资料来源:
\begin{enumerate}
  \item \href{https://rocketmq.apache.org/zh/}{RocketMQ官网}
  \item \href{https://wx.zsxq.com/dweb2/index/footprint/544814425818214}{编程导航星球知识库:Yes大佬的消息队列专栏}
  \item 网上各类与消息队列相关的博客
\end{enumerate}

使用系统:Ubuntu22.04(云服务器)


% 生成目录
\newpage                    %新的一页
\pagenumbering{Roman}       %页码以大写罗马数字形式表示
\setcounter{page}{1}        %设置当前页为第一页
\tableofcontents            %生成目录

% 生成内容
\newpage                    %新的一页
\pagenumbering{arabic}      %页码以阿拉伯数字形式表示
\setcounter{page}{1}        %设置当前页为第一页

% 文章内容
\chapter{初识:消息队列}
\section{介绍}
消息队列,顾名思义,存放\textbf{消息}(可类比为请求)的\textbf{队列}(一种先进先出的数据结构)。

其是一种常用于分布式系统的中间件,可以在不同的应用程序、服务或系统之间传递消息,并且常用于解耦合不同部分的系统,使得系统更加可扩展和灵活。

{\bfseries\kaishu 基本原理:发送者将消息放入队列,接收者从队列中获取消息并处理。}

消息队列实质是一种方式,一种{\bfseries\kaishu 在不同组件之间传递消息的通信方式}。发送者和接收者之间不需要直接通信,它们只需了解如何发送和接收消息即可。

\section{作用与优点}
由上述内容,可推断出消息队列的一些作用:
\begin{itemize}
  \item {\bfseries\kaishu 解耦:}发送者和接收者只需要关心发送消息和接受消息,不用关心彼此。
  \item {\bfseries\kaishu 异步:}发送者不关心消息的处理,即不用等待消息的响应,故支持异步。
  \item {\bfseries\kaishu 削锋:}某些活动的流量过大、请求过多,可能导致系统宕机;消息队列可以作为缓冲区,将这些请求暂时存储起来,以避免瞬时高流量,然后按照系统处理能力逐步消费,实现流量的平滑处理,从而降低系统的压力,避免宕机。
\end{itemize}

以及身为分布式系统的固有优点:
\begin{itemize}
  \item {\bfseries\kaishu 可扩展性:}在解耦后,可方便地单独对发送者或接收者或消息队列进行动态伸缩。
  \item {\bfseries\kaishu 可靠性:}由于消息队列允许多个消费者和生产者,并且通常支持消息持久化和复制,因此即使其中一个组件出现故障,系统仍然可以继续运行并且消息也不会丢失。
\end{itemize}

\section{适用场景}
(真实适用场景还是需要多实践才能掌握,这里仅介绍一些常用场景)
\subsection{异步场景举例:用户注册}
\subsubsection{需求}
用户注册后需向其发送注册邮件和注册短信。

\subsubsection{设计}
用户注册后,将注册信息写入数据库;发送注册邮件;发送短信。

如不使用消息队列,不进行异步解耦,即注册服务器需要同步远程调用写入数据库、发送注册邮件、发送短信的三个函数,将与其他应用发生多次交互,同时还得等待响应,假设一个操作需要0.5s,则该操作会占用注册服务器一个线程的1.5s。

使用消息队列后,注册服务器直接向消息队列中写入三个消息(数据库写入消息、邮件发送消息、短信发送消息),并且是异步发送不用等待返回,假设一次发送消息为0.1s,也仅需0.3s。

\subsection{解耦场景举例:订单-库存管理}
\subsubsection{需求}用户下订单后,库存系统需要减少相对数量。

\subsubsection{设计}
用户下单后,订单系统需要通知库存系统。

\subsubsection{详细设计}
原设计:订单系统调用库存系统的接口。存在缺陷:假如库存系统无法访问,则订单减库存将失败,从而导致订单失败;订单系统依赖库存系统接口,存在耦合。

改进后:订单系统发送订单消息(用户下单后,订单系统完成持久化处理,将消息写入消息队列,返回用户订单下单成功),库存系统读取订单消息并自行处理(订阅订单消息,采用拉/推的方式,获取下单信息,库存系统根据下单信息,进行库存操作)。解决缺陷:假如库存系统无法访问,订单系统仅需要发送消息,可保持运转;订单消息仅发送消息,消息解读由库存系统进行(发布-订阅或消息队列模式),降低耦合度。

\subsection{削锋场景举例:秒杀活动}
\subsubsection{需求}在秒杀活动中,大量用户同时抢购商品,可能会导致系统压力激增。为了应对这一情况,需要一种机制来平稳处理激增的请求流量,避免系统崩溃或性能下降。

\subsubsection{设计}
传统的处理方式可能会导致系统崩溃或性能下降。为了解决这个问题,可以使用消息队列来削峰填谷。

\subsubsection{详细设计}
\begin{enumerate}
  \item 秒杀活动开始:当秒杀活动开始时,用户可以提交秒杀请求。
  \item 请求入队:订单系统接收到用户的秒杀请求后,将请求消息写入消息队列,而不是立即处理。
  \item 消息处理:秒杀请求消息被消息队列按照一定的规则(如先进先出)分发给后端处理程序。
  \item 后端处理:后端处理程序逐条处理消息,检查库存并进行相应的处理(如减少库存、生成订单等)。
\end{enumerate}

以此消息队列可平滑处理激增的请求流量,避免系统因突发流量而崩溃。

\subsection{日志处理场景}
\subsubsection{需求}
需要一种解决大量日志传输和实时处理的方案,以便对日志数据进行分析和可视化展示。

\subsubsection{设计}
设计一个分布式日志处理系统,包括以下组件:
\begin{enumerate}
  \item 日志采集客户端:负责从各个日志源采集日志数据,并将数据定期写入消息队列中。
  \item 消息队列:接收来自日志采集客户端的日志数据,负责数据的存储和转发。
  \item 日志处理应用:订阅并消费Kafka队列中的日志数据,进行实时处理和分析。
  \item Logstash:作为日志处理应用的一部分,负责对原始日志进行解析和转换,统一输出为JSON格式的数据。
  \item Elasticsearch:作为日志处理应用的核心数据存储服务,接收Logstash处理后的JSON格式日志数据,实现实时的数据索引和查询。
  \item Kibana:基于Elasticsearch的数据可视化组件,用于将Elasticsearch中的数据进行可视化展示和分析。
\end{enumerate}

\subsection{消息通讯场景}
\subsubsection{需求}
需要一种高效的消息通讯机制,可以用于点对点通讯或者创建聊天室等场景,以实现实时的消息传递和交流。

\subsubsection{设计}
设计一个基于消息队列的消息通讯系统,包括以下两种场景:
\begin{enumerate}
  \item 点对点通讯:客户端A和客户端B使用同一队列进行消息通讯;消息队列负责接收和转发客户端A和客户端B的消息。
  \item 客户端A、客户端B等多个客户端订阅同一主题:当有客户端发布消息时,消息队列将消息广播给所有订阅了该主题的客户端,客户端收到消息后进行展示。
\end{enumerate}

\section{常用消息队列框架}
\begin{enumerate}
  \item \textbf{RabbitMQ}:RabbitMQ 是一个开源的消息队列系统,实现了高级消息队列协议(AMQP),它是一个可靠、高可用、可扩展的消息代理。RabbitMQ 提供了多种消息传递模式,如点对点、发布/订阅等,适用于各种场景的应用程序。
  \item \textbf{RocketMQ}:RocketMQ 是阿里巴巴开源的分布式消息队列系统,具有高吞吐量、低延迟、高可用性等特点。它支持丰富的消息模型,包括顺序消息、事务消息等,适用于大规模分布式系统的消息通信。
  \item \textbf{Kafka}:Kafka 是由Apache软件基金会开发的分布式流处理平台和消息队列系统。Kafka 设计用于支持大规模的消息处理,具有高吞吐量、持久性、分区等特点,广泛应用于大数据领域。
  \item \textbf{ActiveMQ}:ActiveMQ 是一个开源的消息中间件,实现了 Java Message Service (JMS) 规范。它支持多种传输协议,如TCP、UDP、SSL等,提供了丰富的功能,包括消息持久化、事务支持等。
  \item \textbf{Amazon SQS}:Amazon SQS(Simple Queue Service)是亚马逊提供的消息队列服务,可帮助构建分布式应用程序。它具有高可用性、可扩展性、灵活性等特点,适用于构建在亚马逊云平台上的应用程序。
\end{enumerate}
本文将使用RabbitMQ。

\chapter{启动:RocketMQ}
\section{下载二进制文件包}
官网地址:https://rocketmq.apache.org/zh/docs/quickStart/01quickstart

获得二进制压缩包下载地址:\\https://dist.apache.org/repos/dist/release/rocketmq/5.1.4/rocketmq-all-5.1.4-bin-release.zip

使用wget命令下载压缩包:\\{\bfseries\kaishu wget https://dist.apache.org/repos/dist/release/rocketmq/5.1.4/rocketmq-all-5.1.4-bin-release.zip}
\begin{center}
  \begin{minipage}{\textwidth}
    \center
    \includegraphics[width=\textwidth]{picture/下载二进制文件.png}
    \captionsetup{hypcap=false}
    \captionof{figure}{下载二进制压缩包}
    \label{fig:下载二进制压缩包}
  \end{minipage}
\end{center}

使用unzip命令解压二进制文件压缩包:\\{\bfseries\kaishu unzip rocketmq-all-5.1.4-bin-release.zip}
\begin{center}
  \begin{minipage}{\textwidth}
    \center
    \includegraphics[width=0.7\textwidth]{picture/解压二进制文件.png}
    \captionsetup{hypcap=false}
    \captionof{figure}{解压二进制文件}
    \label{fig:解压二进制文件}
  \end{minipage}
\end{center}

\section{启动NameServer}
进入目录rocketmq-all-5.1.4-bin-release,执行命令:\\{\bfseries\kaishu nohup sh bin/mqnamesrv \&}

命令讲解:
\begin{itemize}
  \item {\bfseries\kaishu nohup:}这代表“不挂起”。在终端中执行命令然后关闭终端时,与该命令相关联的进程通常也会终止。nohup可以防止这种情况发生。
  \item sh:执行脚本文件的shell命令。
  \item bin/mqnamesrv:要运行的脚本路径。
  \item \&:后台运行。
\end{itemize}

发现运行失败:
\begin{center}
  \begin{minipage}{\textwidth}
    \center
    \includegraphics[width=\textwidth]{picture/名字服务器启动失败.png}
    \captionsetup{hypcap=false}
    \captionof{figure}{名字服务器启动失败}
    \label{fig:名字服务器启动失败}
  \end{minipage}
\end{center}

查看nohup.out文件,发现报错:{\bfseries\kaishu OpenJDK 64-Bit Server VM warning: INFO: \\os::commit\_memory(0x0000000700000000, 4294967296, 0) failed; error=\texttt{'}Not enough space\texttt{'} (errno=12)}

操作系统内存不足(由于RocketMQ对内存要求极高,所以自己用云服务器运行基本都会报错),进行修改:

进入rocketmq-all-5.1.4-bin-release/bin目录,对runserver.sh 和 runbroker.sh 以及 tools.sh进行修改。(可使用vim的/+关键字进行查找)
\begin{enumerate}
  \item runserver.sh:
  \\{\bfseries\kaishu JAVA\_OPT=\texttt{"}\${JAVA\_OPT} -server -Xms4g -Xmx4g -Xmn2g -XX:MetaspaceSize=128m 
  \\-XX:MaxMetaspaceSize=320m\texttt{"}}
  \\替换为
  \\{\bfseries\kaishu JAVA\_OPT=\texttt{"}\${JAVA\_OPT} -server -Xms256m -Xmx256m -Xmn128m 
  \\-XX:MetaspaceSize=128m -XX:MaxMetaspaceSize=320m\texttt{"}}
  \\注意有两处。
  \item runbroker.sh:
  \\{\bfseries\kaishu JAVA\_OPT=\texttt{"}\${JAVA\_OPT} -server -Xms8g -Xmx8g\texttt{"}}
  \\替换为{\bfseries\kaishu JAVA\_OPT=\texttt{"}\${JAVA\_OPT} -server -Xms256m -Xmx256m\texttt{"}},
  \\{\bfseries\kaishu JAVA\_OPT=\texttt{"}\${JAVA\_OPT} -Xmn8G -XX:+UseConcMarkSweepGC }
  \\替换为
  \\{\bfseries\kaishu JAVA\_OPT=\texttt{"}\${JAVA\_OPT} -Xmn256m -XX:+UseConcMarkSweepGC }
  \item tools.sh:
  \\{\bfseries\kaishu JAVA\_OPT=\texttt{"}\${JAVA\_OPT} -server -Xms1g -Xmx1g -Xmn256m -XX:MetaspaceSize=128m 
  \\-XX:MaxMetaspaceSize=128m\texttt{"}}
  \\替换为
  \\{\bfseries\kaishu JAVA\_OPT=\texttt{"}\${JAVA\_OPT} -server -Xms256g -Xmx256g -Xmn128m 
  \\-XX:MetaspaceSize=128m -XX:MaxMetaspaceSize=128m\texttt{"}}。
\end{enumerate}

再次输入命令:{\bfseries\kaishu nohup sh bin/mqnamesrv \&}

未报错,查看nohup.out文件,发现启动成功:
\begin{center}
  \begin{minipage}{\textwidth}
    \center
    \includegraphics[width=\textwidth]{picture/名称服务器启动成功.png}
    \captionsetup{hypcap=false}
    \captionof{figure}{名称服务器启动成功}
    \label{fig:名称服务器启动成功}
  \end{minipage}
\end{center}

\section{启动Broker+Proxy}
执行命令:\\{\bfseries\kaishu nohup sh bin/mqbroker -n localhost:9876 --enable-proxy \&}

未报错,查看nohup.out文件,发现启动还是失败:
\begin{center}
  \begin{minipage}{\textwidth}
    \center
    \includegraphics[width=\textwidth]{picture/broker启动失败.png}
    \captionsetup{hypcap=false}
    \captionof{figure}{broker启动失败}
    \label{fig:broker启动失败}
  \end{minipage}
\end{center}

原因是JAVA版本过高,进行修复。

修改runbroker.sh文件:

在{\bfseries\kaishu numactl --interleave=all pwd > /dev/null 2>\&1}上方添加

{\bfseries\kaishu \$JAVA \${JAVA\_OPT} {-}{-}add-exports=java.base/sun.nio.ch=ALL-UNNAMED \$@}


然后再次运行{\bfseries\kaishu nohup sh bin/mqbroker -n localhost:9876 –enable-proxy \&}

并查看nohup.out,发现为不断更新的日志文件,推测运行成功。

查看/root/logs/rocketmqlogs,发现运行成功。
\begin{center}
  \begin{minipage}{\textwidth}
    \center
    \includegraphics[width=\textwidth]{picture/broker启动成功.png}
    \captionsetup{hypcap=false}
    \captionof{figure}{broker启动成功}
    \label{fig:broker启动成功}
  \end{minipage}
\end{center}

至此,RocketMQ启动成功。

% 文章结束
\end{document}